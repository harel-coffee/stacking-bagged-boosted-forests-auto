\documentclass[]{article}
\usepackage{lmodern}
\usepackage{amssymb,amsmath}
\usepackage{ifxetex,ifluatex}
\usepackage{fixltx2e} % provides \textsubscript
\ifnum 0\ifxetex 1\fi\ifluatex 1\fi=0 % if pdftex
  \usepackage[T1]{fontenc}
  \usepackage[utf8]{inputenc}
\else % if luatex or xelatex
  \ifxetex
    \usepackage{mathspec}
    \usepackage{xltxtra,xunicode}
  \else
    \usepackage{fontspec}
  \fi
  \defaultfontfeatures{Mapping=tex-text,Scale=MatchLowercase}
  \newcommand{\euro}{€}
\fi
% use upquote if available, for straight quotes in verbatim environments
\IfFileExists{upquote.sty}{\usepackage{upquote}}{}
% use microtype if available
\IfFileExists{microtype.sty}{%
\usepackage{microtype}
\UseMicrotypeSet[protrusion]{basicmath} % disable protrusion for tt fonts
}{}
\usepackage[margin=1in]{geometry}
\usepackage{graphicx}
\makeatletter
\def\maxwidth{\ifdim\Gin@nat@width>\linewidth\linewidth\else\Gin@nat@width\fi}
\def\maxheight{\ifdim\Gin@nat@height>\textheight\textheight\else\Gin@nat@height\fi}
\makeatother
% Scale images if necessary, so that they will not overflow the page
% margins by default, and it is still possible to overwrite the defaults
% using explicit options in \includegraphics[width, height, ...]{}
\setkeys{Gin}{width=\maxwidth,height=\maxheight,keepaspectratio}
\ifxetex
  \usepackage[setpagesize=false, % page size defined by xetex
              unicode=false, % unicode breaks when used with xetex
              xetex]{hyperref}
\else
  \usepackage[unicode=true]{hyperref}
\fi
\hypersetup{breaklinks=true,
            bookmarks=true,
            pdfauthor={Raphael Rodrigues Campos},
            pdftitle={Effectivess comparison report},
            colorlinks=true,
            citecolor=blue,
            urlcolor=blue,
            linkcolor=magenta,
            pdfborder={0 0 0}}
\urlstyle{same}  % don't use monospace font for urls
\setlength{\parindent}{0pt}
\setlength{\parskip}{6pt plus 2pt minus 1pt}
\setlength{\emergencystretch}{3em}  % prevent overfull lines
\setcounter{secnumdepth}{0}

%%% Use protect on footnotes to avoid problems with footnotes in titles
\let\rmarkdownfootnote\footnote%
\def\footnote{\protect\rmarkdownfootnote}

%%% Change title format to be more compact
\usepackage{titling}

% Create subtitle command for use in maketitle
\newcommand{\subtitle}[1]{
  \posttitle{
    \begin{center}\large#1\end{center}
    }
}

\setlength{\droptitle}{-2em}
  \title{Effectivess comparison report}
  \pretitle{\vspace{\droptitle}\centering\huge}
  \posttitle{\par}
  \author{Raphael Rodrigues Campos}
  \preauthor{\centering\large\emph}
  \postauthor{\par}
  \predate{\centering\large\emph}
  \postdate{\par}
  \date{January 17, 2016}

\usepackage{multirow}


\begin{document}

\maketitle


Eu implementei o BROOF usando Extremely Randomized Trees no lugar da RF,
gerando o algoritmo que chamei de BERT (Boosted Extremely Randomized
Trees).

A própria ERT se sai melhor em alguns datasets do que a RF. Portanto,
era de se esperar que a BERT se saísse um pouco melhor que o BROOF, como
pode-se verificar no arquivo anexo.

O arquivo anexo possui uma tabela comparando todos os métodos rodados
até agora.

Além da implementaćão do BERT, eu também implementei método de ensemble
``Stacked Generalization'' descrito em {[}1{]} David H. Wolpert,
``Stacked Generalization'', Neural Networks, 5, 241--259, 1992.

O método comb1 na tabela é o stacking de 2 níveis para combinaćão dos
métodos LazyNN\_RF e BROOF. No nível do zero do stacking foram
utilizados os classificadores LazyNN\_RF e BROOF para gerar o conjunto
de treino do nível 1. No nível 1 foi utilizado uma RF com 200 árvores.

Os resultados apresentados são promissores. Sobretudo quando se trata de
métrica microf1, onde tivemos mais ganhos significativos.

\section{Resultados}\label{resultados}

\% latex table generated in R 3.2.4 by xtable 1.8-0 package \% Sun Apr
17 12:53:10 2016

\begin{table}[ht]
\centering
\begin{tabular}{llllll}
  \hline
V1 & V2 & 20NG & 4UNI & ACM & REUTERS90 \\ 
  \cline{3-6} \hline
\multirow{2}{*}{BERT} & microF1 & 88.93 $\pm$  0.39 & 84.61 $\pm$  0.98 & 74.8 $\pm$  0.59 & 67.33 $\pm$  0.72 \\ 
   & macroF1 & 88.59 $\pm$  0.5 & 73.61 $\pm$  1.85 & \bf{62.1 $\pm$  0.99} & 29.24 $\pm$  1.4 \\ 
   \cline{3-6}\multirow{2}{*}{BROOF} & microF1 & 87.96 $\pm$  0.24 & 84.41 $\pm$  1.07 & 73.35 $\pm$  0.79 & 66.79 $\pm$  0.97 \\ 
   & macroF1 & 87.44 $\pm$  0.28 & 73.23 $\pm$  1.1 & 60.76 $\pm$  0.8 & 28.48 $\pm$  2.17 \\ 
   \cline{3-6}\multirow{2}{*}{COMB1} & microF1 & 89.32 $\pm$  0.42 & \bf{86.52 $\pm$  1.18} & 76.74 $\pm$  0.73 & \bf{77.22 $\pm$  1.14} \\ 
   & macroF1 & 89.01 $\pm$  0.44 & \bf{78.66 $\pm$  1.9} & \bf{62.2 $\pm$  1.01} & \bf{31.71 $\pm$  2.7} \\ 
   \cline{3-6}\multirow{2}{*}{COMB2} & microF1 & 90.2 $\pm$  0.51 & \bf{86.54 $\pm$  1.06} & 76.88 $\pm$  0.55 & \bf{78.25 $\pm$  1.17} \\ 
   & macroF1 & 89.95 $\pm$  0.52 & \bf{79.41 $\pm$  1.63} & \bf{62.66 $\pm$  0.81} & \bf{32.86 $\pm$  2.23} \\ 
   \cline{3-6}\multirow{2}{*}{COMB3} & microF1 & 90.63 $\pm$  0.57 & \bf{86.79 $\pm$  0.86} & \bf{77.34 $\pm$  0.6} & \bf{79 $\pm$  1.14} \\ 
   & macroF1 & 90.4 $\pm$  0.57 & \bf{79.63 $\pm$  1.91} & \bf{62.91 $\pm$  0.92} & \bf{33.93 $\pm$  2.97} \\ 
   \cline{3-6}\multirow{2}{*}{COMBALL} & microF1 & \bf{91.67 $\pm$  0.44} & \bf{86.74 $\pm$  1.17} & \bf{78.46 $\pm$  0.72} & 0 $\pm$  0 \\ 
   & macroF1 & \bf{91.43 $\pm$  0.42} & \bf{79.45 $\pm$  2.23} & \bf{63.72 $\pm$  1.01} & 0 $\pm$  0 \\ 
   \cline{3-6}\multirow{2}{*}{COMBSOTA} & microF1 & 90.65 $\pm$  0.4 & 83.79 $\pm$  1.3 & \bf{77.9 $\pm$  0.73} & 74.41 $\pm$  1.21 \\ 
   & macroF1 & 90.41 $\pm$  0.4 & 74.19 $\pm$  2.13 & \bf{63.15 $\pm$  0.76} & 28.18 $\pm$  1.58 \\ 
   \cline{3-6}\multirow{2}{*}{KNN} & microF1 & 87.53 $\pm$  0.69 & 75.63 $\pm$  0.94 & 70.99 $\pm$  0.96 & 68.07 $\pm$  1.07 \\ 
   & macroF1 & 87.22 $\pm$  0.66 & 60.34 $\pm$  1.36 & 55.85 $\pm$  0.97 & \bf{29.93 $\pm$  2.48} \\ 
   \cline{3-6}\multirow{2}{*}{LAZY} & microF1 & 87.96 $\pm$  0.37 & 82.34 $\pm$  0.61 & 74.02 $\pm$  0.79 & 66.3 $\pm$  1.07 \\ 
   & macroF1 & 87.39 $\pm$  0.37 & 68.33 $\pm$  1.6 & 59.46 $\pm$  1.35 & 26.61 $\pm$  2.12 \\ 
   \cline{3-6}\multirow{2}{*}{LXT} & microF1 & 88.39 $\pm$  0.51 & 81.24 $\pm$  0.71 & 69.63 $\pm$  0.91 & 65.92 $\pm$  0.82 \\ 
   & macroF1 & 88.05 $\pm$  0.44 & 66.89 $\pm$  1.23 & 57.33 $\pm$  1.48 & 26.71 $\pm$  2.53 \\ 
   \cline{3-6}\multirow{2}{*}{NB} & microF1 & 88.99 $\pm$  0.54 & 62.63 $\pm$  1.7 & 73.54 $\pm$  0.71 & 65.32 $\pm$  1.13 \\ 
   & macroF1 & 88.68 $\pm$  0.55 & 51.38 $\pm$  3.19 & 58.03 $\pm$  0.85 & 27.86 $\pm$  0.79 \\ 
   \cline{3-6}\multirow{2}{*}{RF} & microF1 & 83.64 $\pm$  0.29 & 81.52 $\pm$  1 & 71.05 $\pm$  0.31 & 63.92 $\pm$  0.81 \\ 
   & macroF1 & 83.08 $\pm$  0.35 & 65.44 $\pm$  1.91 & 56.56 $\pm$  0.45 & 24.36 $\pm$  1.98 \\ 
   \cline{3-6}\multirow{2}{*}{SVM} & microF1 & 88.35 $\pm$  0.37 & 81.36 $\pm$  1.01 & 73.82 $\pm$  0.78 & 67.6 $\pm$  1.1 \\ 
   & macroF1 & 88.3 $\pm$  0.38 & 68.01 $\pm$  2.39 & \bf{62.55 $\pm$  1.53} & \bf{31.73 $\pm$  3.13} \\ 
   \cline{3-6}\multirow{2}{*}{XT} & microF1 & 85.94 $\pm$  0.23 & 81.66 $\pm$  1.03 & 71.94 $\pm$  0.66 & 64.33 $\pm$  0.86 \\ 
   & macroF1 & 85.57 $\pm$  0.22 & 65.44 $\pm$  2.41 & 57.4 $\pm$  1.13 & 24.47 $\pm$  2.22 \\ 
   \cline{3-6}\multirow{2}{*}{XT2} & microF1 & 85.94 $\pm$  0.23 & 0 $\pm$  0 & 0 $\pm$  0 & 0 $\pm$  0 \\ 
   & macroF1 & 85.57 $\pm$  0.22 & 0 $\pm$  0 & 0 $\pm$  0 & 0 $\pm$  0 \\ 
   \cline{3-6}\end{tabular}
\caption{Comparaćão entre todos os métodos} 
\end{table}

Legenda para os métodos:

\begin{itemize}
\itemsep1pt\parskip0pt\parsep0pt
\item
  BERT: Boosted Extremely Randomized Trees
\item
  LXT: Lazy Extremely Randomized Trees
\item
  RF: Random Forest com 200 árvores
\item
  RF1000: Random Forest com 1000 árvores
\item
  XT: Extremely Randomized Trees com 200 árvores
\item
  XT1000: Extremely Randomized Trees com 1000 árvores
\item
  COMB1: Stacking (Lazy + BROOF)
\item
  COMB2: Stacking (LXT + BERT)
\item
  COMB3: Stacking (Lazy + BROOF + LXT + BERT)
\item
  COMBSOTA: Stacking (KNN + RF + SVM + NB)
\end{itemize}

\end{document}
