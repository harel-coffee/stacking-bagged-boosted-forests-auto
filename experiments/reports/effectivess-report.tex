\documentclass[]{article}
\usepackage{lmodern}
\usepackage{amssymb,amsmath}
\usepackage{ifxetex,ifluatex}
\usepackage{fixltx2e} % provides \textsubscript
\ifnum 0\ifxetex 1\fi\ifluatex 1\fi=0 % if pdftex
  \usepackage[T1]{fontenc}
  \usepackage[utf8]{inputenc}
\else % if luatex or xelatex
  \ifxetex
    \usepackage{mathspec}
    \usepackage{xltxtra,xunicode}
  \else
    \usepackage{fontspec}
  \fi
  \defaultfontfeatures{Mapping=tex-text,Scale=MatchLowercase}
  \newcommand{\euro}{€}
\fi
% use upquote if available, for straight quotes in verbatim environments
\IfFileExists{upquote.sty}{\usepackage{upquote}}{}
% use microtype if available
\IfFileExists{microtype.sty}{%
\usepackage{microtype}
\UseMicrotypeSet[protrusion]{basicmath} % disable protrusion for tt fonts
}{}
\usepackage[margin=1in]{geometry}
\usepackage{graphicx}
\makeatletter
\def\maxwidth{\ifdim\Gin@nat@width>\linewidth\linewidth\else\Gin@nat@width\fi}
\def\maxheight{\ifdim\Gin@nat@height>\textheight\textheight\else\Gin@nat@height\fi}
\makeatother
% Scale images if necessary, so that they will not overflow the page
% margins by default, and it is still possible to overwrite the defaults
% using explicit options in \includegraphics[width, height, ...]{}
\setkeys{Gin}{width=\maxwidth,height=\maxheight,keepaspectratio}
\ifxetex
  \usepackage[setpagesize=false, % page size defined by xetex
              unicode=false, % unicode breaks when used with xetex
              xetex]{hyperref}
\else
  \usepackage[unicode=true]{hyperref}
\fi
\hypersetup{breaklinks=true,
            bookmarks=true,
            pdfauthor={Raphael Rodrigues Campos},
            pdftitle={Effectivess comparison report},
            colorlinks=true,
            citecolor=blue,
            urlcolor=blue,
            linkcolor=magenta,
            pdfborder={0 0 0}}
\urlstyle{same}  % don't use monospace font for urls
\setlength{\parindent}{0pt}
\setlength{\parskip}{6pt plus 2pt minus 1pt}
\setlength{\emergencystretch}{3em}  % prevent overfull lines
\setcounter{secnumdepth}{0}

%%% Use protect on footnotes to avoid problems with footnotes in titles
\let\rmarkdownfootnote\footnote%
\def\footnote{\protect\rmarkdownfootnote}

%%% Change title format to be more compact
\usepackage{titling}

% Create subtitle command for use in maketitle
\newcommand{\subtitle}[1]{
  \posttitle{
    \begin{center}\large#1\end{center}
    }
}

\setlength{\droptitle}{-2em}
  \title{Effectivess comparison report}
  \pretitle{\vspace{\droptitle}\centering\huge}
  \posttitle{\par}
  \author{Raphael Rodrigues Campos}
  \preauthor{\centering\large\emph}
  \postauthor{\par}
  \predate{\centering\large\emph}
  \postdate{\par}
  \date{January 17, 2016}

\usepackage{multirow}


\begin{document}

\maketitle


Eu implementei o BROOF usando Extremely Randomized Trees no lugar da RF,
gerando o algoritmo que chamei de BERT (Boosted Extremely Randomized
Trees).

A própria ERT se sai melhor em alguns datasets do que a RF. Portanto,
era de se esperar que a BERT se saísse um pouco melhor que o BROOF, como
pode-se verificar no arquivo anexo.

O arquivo anexo possui uma tabela comparando todos os métodos rodados
até agora.

Além da implementaćão do BERT, eu também implementei método de ensemble
``Stacked Generalization'' descrito em {[}1{]} David H. Wolpert,
``Stacked Generalization'', Neural Networks, 5, 241--259, 1992.

O método comb1 na tabela é o stacking de 2 níveis para combinaćão dos
métodos LazyNN\_RF e BROOF. No nível do zero do stacking foram
utilizados os classificadores LazyNN\_RF e BROOF para gerar o conjunto
de treino do nível 1. No nível 1 foi utilizado uma RF com 200 árvores.

Os resultados apresentados são promissores. Sobretudo quando se trata de
métrica microf1, onde tivemos mais ganhos significativos.

\section{Resultados}\label{resultados}

\% latex table generated in R 3.2.3 by xtable 1.8-0 package \% Wed Mar 9
22:21:12 2016

\begin{table}[ht]
\centering
\begin{tabular}{llllll}
  \hline
V1 & V2 & 20NG & 4UNI & ACM & REUTERS90 \\ 
  \hline
\multirow{2}{*}{BERT} & microF1 & 89.13 $\pm$  0.41 & \bf{84.53 $\pm$  0.9} & 74.66 $\pm$  0.63 & 67.23 $\pm$  0.86 \\ 
   & macroF1 & 88.8 $\pm$  0.52 & 72.64 $\pm$  1.96 & 61.83 $\pm$  0.98 & 29.27 $\pm$  2.26 \\ 
  \multirow{2}{*}{BROOF} & microF1 & 87.56 $\pm$  0.23 & \bf{84.42 $\pm$  0.7} & 73.25 $\pm$  0.69 & 66.48 $\pm$  0.9 \\ 
   & macroF1 & 87.06 $\pm$  0.18 & 73.64 $\pm$  0.95 & 60.01 $\pm$  0.94 & 28.76 $\pm$  2.65 \\ 
  \multirow{2}{*}{COMB1} & microF1 & \bf{89.74 $\pm$  0.57} & \bf{86.4 $\pm$  0.91} & \bf{77.05 $\pm$  0.64} & \bf{77.99 $\pm$  1.33} \\ 
   & macroF1 & \bf{89.46 $\pm$  0.61} & \bf{78.08 $\pm$  1.8} & 62.8 $\pm$  0.88 & \bf{34.12 $\pm$  3.7} \\ 
  \multirow{2}{*}{COMB3} & microF1 & \bf{90.71 $\pm$  0.39} & \bf{86.44 $\pm$  1.17} & \bf{77.86 $\pm$  0.98} & 0 $\pm$  0 \\ 
   & macroF1 & \bf{90.49 $\pm$  0.36} & \bf{78.23 $\pm$  1.9} & \bf{63.55 $\pm$  1.09} & 0 $\pm$  0 \\ 
  \multirow{2}{*}{COMBSOTA} & microF1 & \bf{89.9 $\pm$  0.41} & 83.38 $\pm$  0.97 & \bf{76.61 $\pm$  0.84} & 73.94 $\pm$  1.16 \\ 
   & macroF1 & \bf{89.63 $\pm$  0.42} & 70.36 $\pm$  2.87 & \bf{64.73 $\pm$  1.68} & \bf{31.53 $\pm$  1.56} \\ 
  \multirow{2}{*}{KNN} & microF1 & 87.41 $\pm$  0.7 & 75.02 $\pm$  1.39 & 70.41 $\pm$  0.81 & 69.04 $\pm$  0.96 \\ 
   & macroF1 & 87.11 $\pm$  0.68 & 60.08 $\pm$  1.12 & 59.72 $\pm$  0.96 & \bf{35.35 $\pm$  1.43} \\ 
  \multirow{2}{*}{LAZY} & microF1 & 88.22 $\pm$  0.29 & 82.04 $\pm$  0.83 & 73.41 $\pm$  0.79 & 65.72 $\pm$  1.06 \\ 
   & macroF1 & 87.75 $\pm$  0.35 & 67.77 $\pm$  1.2 & 61.56 $\pm$  1.69 & 26.85 $\pm$  2.92 \\ 
  \multirow{2}{*}{LXT} & microF1 & 88.49 $\pm$  0.43 & 82.15 $\pm$  0.81 & 71.71 $\pm$  0.69 & 65.82 $\pm$  1.25 \\ 
   & macroF1 & 88.19 $\pm$  0.39 & 68.1 $\pm$  1.72 & 60.32 $\pm$  0.55 & 28.73 $\pm$  2.95 \\ 
  \multirow{2}{*}{NB} & microF1 & 88.99 $\pm$  0.54 & 59.76 $\pm$  1.75 & 71.79 $\pm$  1.01 & 64.86 $\pm$  1.59 \\ 
   & macroF1 & 88.68 $\pm$  0.55 & 53.96 $\pm$  1.28 & 57.59 $\pm$  0.51 & 26.76 $\pm$  1.54 \\ 
  \multirow{2}{*}{RF1000} & microF1 & 86.49 $\pm$  0.46 & 81.37 $\pm$  0.85 & 71.41 $\pm$  0.53 & 63.88 $\pm$  0.96 \\ 
   & macroF1 & 85.93 $\pm$  0.49 & 66.7 $\pm$  1.52 & 56.78 $\pm$  0.49 & 24.8 $\pm$  2.29 \\ 
  \multirow{2}{*}{RF} & microF1 & 84.03 $\pm$  0.39 & 81.25 $\pm$  1.13 & 71.06 $\pm$  0.48 & 63.83 $\pm$  1.13 \\ 
   & macroF1 & 83.55 $\pm$  0.38 & 66.9 $\pm$  1.9 & 56.37 $\pm$  0.58 & 24.51 $\pm$  1.91 \\ 
  \multirow{2}{*}{SVM} & microF1 & \bf{90.77 $\pm$  0.49} & 83.36 $\pm$  0.93 & 76.05 $\pm$  0.61 & 68.08 $\pm$  1.06 \\ 
   & macroF1 & \bf{90.53 $\pm$  0.48} & 71.89 $\pm$  2.54 & \bf{65.69 $\pm$  1.14} & \bf{33.02 $\pm$  2.57} \\ 
  \multirow{2}{*}{XT1000} & microF1 & 88.71 $\pm$  0.52 & 82.61 $\pm$  1 & 73.53 $\pm$  0.69 & 64.87 $\pm$  0.95 \\ 
   & macroF1 & 88.32 $\pm$  0.64 & 66.55 $\pm$  2.02 & 59.11 $\pm$  0.83 & 25.45 $\pm$  2.6 \\ 
  \multirow{2}{*}{XT} & microF1 & 86.83 $\pm$  0.49 & 82.49 $\pm$  1.07 & 73.15 $\pm$  0.68 & 64.89 $\pm$  1.01 \\ 
   & macroF1 & 86.49 $\pm$  0.51 & 66.76 $\pm$  2.12 & 58.93 $\pm$  0.91 & 25.36 $\pm$  2.81 \\ 
   \hline
\end{tabular}
\caption{Comparaćão entre todos os métodos} 
\end{table}

Legenda para os métodos:

\begin{itemize}
\itemsep1pt\parskip0pt\parsep0pt
\item
  BERT: Boosted Extremely Randomized Trees
\item
  LXT: Lazy Extremely Randomized Trees
\item
  RF: Random Forest com 200 árvores
\item
  RF1000: Random Forest com 1000 árvores
\item
  XT: Extremely Randomized Trees com 200 árvores
\item
  XT1000: Extremely Randomized Trees com 1000 árvores
\item
  COMB1: Stacking (Lazy + BROOF)
\item
  COMB2: Stacking (LXT + BERT)
\item
  COMB3: Stacking (Lazy + BROOF + LXT + BERT)
\item
  COMBSOTA: Stacking (KNN + RF + SVM + NB)
\end{itemize}

\end{document}
