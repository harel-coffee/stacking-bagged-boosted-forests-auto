\documentclass[]{article}
\usepackage{lmodern}
\usepackage{amssymb,amsmath}
\usepackage{ifxetex,ifluatex}
\usepackage{fixltx2e} % provides \textsubscript
\ifnum 0\ifxetex 1\fi\ifluatex 1\fi=0 % if pdftex
  \usepackage[T1]{fontenc}
  \usepackage[utf8]{inputenc}
\else % if luatex or xelatex
  \ifxetex
    \usepackage{mathspec}
    \usepackage{xltxtra,xunicode}
  \else
    \usepackage{fontspec}
  \fi
  \defaultfontfeatures{Mapping=tex-text,Scale=MatchLowercase}
  \newcommand{\euro}{€}
\fi
% use upquote if available, for straight quotes in verbatim environments
\IfFileExists{upquote.sty}{\usepackage{upquote}}{}
% use microtype if available
\IfFileExists{microtype.sty}{%
\usepackage{microtype}
\UseMicrotypeSet[protrusion]{basicmath} % disable protrusion for tt fonts
}{}
\usepackage[margin=1in]{geometry}
\usepackage{graphicx}
\makeatletter
\def\maxwidth{\ifdim\Gin@nat@width>\linewidth\linewidth\else\Gin@nat@width\fi}
\def\maxheight{\ifdim\Gin@nat@height>\textheight\textheight\else\Gin@nat@height\fi}
\makeatother
% Scale images if necessary, so that they will not overflow the page
% margins by default, and it is still possible to overwrite the defaults
% using explicit options in \includegraphics[width, height, ...]{}
\setkeys{Gin}{width=\maxwidth,height=\maxheight,keepaspectratio}
\ifxetex
  \usepackage[setpagesize=false, % page size defined by xetex
              unicode=false, % unicode breaks when used with xetex
              xetex]{hyperref}
\else
  \usepackage[unicode=true]{hyperref}
\fi
\hypersetup{breaklinks=true,
            bookmarks=true,
            pdfauthor={Raphael Rodrigues Campos},
            pdftitle={Stacking utilizado nos experimentos},
            colorlinks=true,
            citecolor=blue,
            urlcolor=blue,
            linkcolor=magenta,
            pdfborder={0 0 0}}
\urlstyle{same}  % don't use monospace font for urls
\setlength{\parindent}{0pt}
\setlength{\parskip}{6pt plus 2pt minus 1pt}
\setlength{\emergencystretch}{3em}  % prevent overfull lines
\setcounter{secnumdepth}{0}

%%% Use protect on footnotes to avoid problems with footnotes in titles
\let\rmarkdownfootnote\footnote%
\def\footnote{\protect\rmarkdownfootnote}

%%% Change title format to be more compact
\usepackage{titling}

% Create subtitle command for use in maketitle
\newcommand{\subtitle}[1]{
  \posttitle{
    \begin{center}\large#1\end{center}
    }
}

\setlength{\droptitle}{-2em}
  \title{Stacking utilizado nos experimentos}
  \pretitle{\vspace{\droptitle}\centering\huge}
  \posttitle{\par}
  \author{Raphael Rodrigues Campos}
  \preauthor{\centering\large\emph}
  \postauthor{\par}
  \predate{\centering\large\emph}
  \postdate{\par}
  \date{28 abril, 2016}

\usepackage{multirow} \usepackage[utf8]{inputenc} \usepackage{mathtools}


\begin{document}

\maketitle


\subsection{Stacking}\label{stacking}

Stacking também conhecido como ``Stacked Generalization'' é um método
para combinar multiplos classificadores usando algoritmos de
aprendizados heterogêneos \(L_1, ..., L_N\) sobre um único conjunto de
dados \(D\), que consiste de exemplos \(e_i = (x_i, y_i)\), onde \(x_i\)
é o vetor de atributos e \(y_i\) sua classificação.

\subsubsection{Stacking Framework}\label{stacking-framework}

O staking framework utilizado é baseado no descrito em {[}1{]} David H.
Wolpert, ``Stacked Generalization'', Neural Networks, 5, 241--259, 1992.
Foi utilizado um stacking de dois níveis (o framework não se limita a
apenas dois níveis, é possível fazer o stacking de quantos níveis julgar
necessário), que pode ser dividido em duas fases. Na primeira fase, um
conjunto de classificadores do nível base \(C_1, C_2, ..., C_N\) é
gerado, onde \(C_i = L_i(D)\). Na segunda fase um classidicador do
meta-nível aprende a combinar as saídas dos classificadores do nível
base.

Para gerar o conjunto de treino para o aprendizado do classificador do
meta-nível, pode-se aplicar o procedimento \textbf{leave-one-out} ou
\textbf{cross validation}. Por questões óbvias de custo computacional, é
utilizado nesse relatório cross validation, mais especificamente
\textbf{5-fold cross validation}. Cada classificador do nível base
aprende usando \(D - F_k\) deixando o k-ésimo \emph{fold} para teste:
\(\forall i = 1,...,N : \forall k = 1,...,5 : C^{k}_i = L_i(D-F_k)\).
Agora, os classificadores recém aprendidos são usados para gerar as
predições para \(\forall x_j \in F_k:\hat{y}_j^i=C^k_i(x_j)\). O
conjunto de treino do meta-nível consiste de exemplos da seguinte forma
\(((\hat{y}_i^1,..., \hat{y}_i^N), y_i)\), onde os atributos são as
predicóes do s classificadores do nível base e a classe é a classe
correta sabida de antemão.

\paragraph{Exemplo}\label{exemplo}

Esse procedimento pode parecer complicado, mas na verdade é simples.
Como um exemplo, vamos gerar alguns dados sintéticos com a função
``saída = soma do três componentes de entrada''. Nosso conjunto de
treino D consiste de 5 pares de entrada e saída
\(\{((0,0,0),0), ((1,0,0),1), ((1,2,0),3), ((1,1,1),3), ((1,-2,4),3)\}\),
todas as entradas sem ruídos. Vamos rotular esses 5 pares de entrada e
saída como \(F_1\) até \(F_5\) (Então por exemplo \(D - F_2\) consiste
dos quatros pares
\(\{((0,0,0),0), ((1,2,0),3), ((1,1,1),3), ((1,-2,4),3)\}\)). Nesse
exemplo, temos dois classificadores do nível base \(C_1\) e \(C_2\), e
um único classificador do meta-nível \(\Gamma\). O conjunto de treino do
meta-nível \(D'\) é dado pelo cinco pares de entrada e saída \(\{\)
((\(C_1^k(F_k), C_2^k(F_k)\)), componente de saída de
\(F_k) : \forall k \in \{1,...,5\}\) e \(C_i^k = L_i(D-F_k)\}\) (Esse
espaço do meta-nível possui duas dimensões de entrada e uma de saída).
Ou seja, a instância do conjunto de treino do meta-nível correspondente
a \(k = 1\) tem o componetne de saída 0 e entrada
\((C_1^1((0,0,0)), C_2^1((0,0,0)))\). Agora nos é dado um exemplo de
teste no formato do nível base \((x_1, x_2, x_3)\). Nós predizemos seu
valor com \(\Gamma((C_1((x_1, x_2, x_3)), (C_2((x_1, x_2, x_3)))\), onde
\(C_1\) e \(C_2\) foram treinados com todo \(D\), e \(\Gamma\) com
\(D'\). Em outras palavras, nós predizemos o valor da entrada de teste
\(q = (x_1, x_2, x_3)\) treinando \(\Gamma\) em \(D'\) e assim
predizendo a entrada formada pelas predições do valor do exemplo de
teste \(q\), de ambos classificadores do nível base \(C_1\) e \(C_2\),
que por suas vezes foram treinados com todo \(D\).

\subsubsection{Stacking com distribuições de
probabilidade}\label{stacking-com-distribuicoes-de-probabilidade}

Usar probabilidade para gerar o conjunto de treino do meta-nível é mais
vantajoso já que disponibiliza mais informação acerca das predições
feitas pelos classificadores do nível base. Essa informações adicionais
permitem que não seja usado somente a predição, mas também o confiança
de cada classificador do nível base.

Nessa abordagem, cada classificador do nível base prediz uma
Distribuição de Probabilidade (DP) sobre todas as classes possíveis.
Então, a predição do classificador do nível base \(C\) apliacado a um
exemplo \(x\) é a DP: \(p^C(x) = (p^C(c_1|x), ... , p^C(c_m|x))\), onde
\(\{c_1, ..., c_m\}\) é o conjunto de possíveis valores para as classes
e \(p^C(c_i|x))\) descreve a probabilidade do exemplo \(x\) ser da
classe \(c_i\) estimado pelo classificador \(C\). A classe \(c_j\) com
maior probalidade será classe predita por \(C\). Dessa forma, os
atributos do meta-nível serão as probabilidade preditas para cada classe
possível por cada classificador do nível base. O número total de
atributos no conjunto de treino do meta-nível seria \(Nm\), \(m\)
atributos para cada classificador do nível base.

Os experimentos rodados até então utilizaram o stacking framework com
DPs.

\subsubsection{Stacking com DP, Entropia e probabilidade
máxima}\label{stacking-com-dp-entropia-e-probabilidade-maxima}

No artigo {[}2{]} Is combining classifiers better than selecting the
best one, os autores propões uma extensão para esse framework com DP
espandindo o número de meta-atributos. Esse novos meta-atributos seriam:

\begin{itemize}
\item
  A distribuição de probabilidade mutiplicadao pela probabilidade
  máxima:
  \(p_{C_j} = p^{C_j}(c_i|x) \times M_{C_j}(x) = p^{C_j}(c_i|x) \times max_{i=1}^{m}(p^{C_j}(c_i|x))\),
  \(\forall i \in \{1,...,m\}\) e \(\forall j \in \{1,...,N\}\).
\item
  As entropias das distripuições de probabilidade:
  \(E_{C_j}(x) = -\sum_{i=1}^{m}p^{C_j}(c_i|x).\log_2(p^{C_j}(c_i|x))\).
\end{itemize}

O número total de atributos do meta-nível é \(N(2m+1)\).

A idea é obter ainda mais informações em relação a predição feita pelos
classificadores do nível base. Como Ting and Witten (1999) disseram: o
uso de distribuição de probabilidades tem a vantagemde capturar não
apenas as predições dos classificadores do nível base, mas também, suas
certezas. Os atributos adicionais tentam capturar a certeza de forma
mais explicita.

Entropia é uma medida de incerteza. Quanto maior a entropia da
distribuição menor é a certeza sobre a predição. A probabilidade máxima
de uma DP \(M_{C_j}\) também contém informação sobre certeza da
predição: quanto maior \(M_{C_j}\) for mais certo daquela resposta o
classificador do nível base está, e vice versa.

Esse é uma ideia para aplicarmos futuramente. Nesse momento continuarei
utilizanto somente a DP.

\end{document}
